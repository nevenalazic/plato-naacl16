
\begin{figure}[t!]
\label{fig:model}
\begin{tikzpicture}[node distance=0.8cm and 1cm,>=stealth',auto]
\tikzstyle{candidate}=[circle,draw=black,fill=gray!10,font=\large]
  \tikzstyle{section}=[rectangle,minimum size=6mm,font=\large]
\begin{scope}[-,thick]

  \node[candidate]  (x1)      {$y_1$} ;
  \node[candidate, right=of x1] (x2) {$y_2$} ; 
  \node[candidate, right=of x2] (x3) {$y_3$} ; 
  \node[candidate, right=of x3] (x4) {$y_4$} ; 
  \node[section, left=0.1cm of x1] (a) {(a)};
  \path(x1.east) edge (x2.west);
  \path(x2.east) edge (x3.west);
  \path(x3.east) edge (x4.west);
  \path(x1.east) edge [bend left=45] (x3.west);
  \path(x2.east) edge [bend left=45] (x4.west);
    \path(x1.east) edge [bend left=45] (x4.west);

  \node[candidate, below=of x1]  (y1)      {$y_1$} ;
  \node[candidate, right=of y1] (y2) {$y_2$} ; 
  \node[candidate, right=of y2] (y3) {$y_3$} ; 
  \node[candidate, right=of y3] (y4) {$y_4$} ; 
    \node[section, left=0.1cm of y1] (b) {(b)};
  \path(y1.east) edge (y2.west);
  \path(y2.east) edge (y3.west);
  \path(y2.east) edge [bend left=45] (y4.west);
  
    \node[candidate, below=of y1]  (z1)      {$y_1$} ;
  \node[candidate, right=of z1] (z2) {$y_2$} ; 
  \node[candidate, right=of z2] (z3) {$y_3$} ; 
  \node[candidate, right=of z3] (z4) {$y_4$} ; 
    \node[section, left=0.1cm of z1] (c) {(c)};
  \path(z2.east) edge (z3.west);
  \path(z3.east) edge (z4.west);
  \path(z1.east) edge [bend left=45] (z3.west);
\end{scope}
\end{tikzpicture}
\caption{(a) The complete graph corresponding to \eqref{eq:global_obj}. (b) A star shaped subgraph corresponding to $y_2$. This will be used to obtaining the label $y_2$. (c) The star graph for $y_3$.}
\label{fig:star}
\end{figure}