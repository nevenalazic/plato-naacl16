\section{Definitions and notation}
\label{sec:notation}

We are given a document with $n$ mentions, where each mention $i$ has a set of $n_i$ candidate entities $\mathcal{C}_i = \{c_{i,1}, ..., c_{i,n_i}\}$. The goal is to assign a label $y_i \in \mathcal{C}_i$ to each mention.

Similarly to previous work, our approach to disambiguation relies on local and pairwise candidate scores, which we denote by $s_i(y_i)$ and $s_{ij}(y_i, y_j)$ respectively. The local score is based only on local evidence such as the mention phrase and textual features, while the pairwise score is based on the relatedness of the two candidates. 
In Sections \ref{sec:score_param} and \ref{sec:learning} we discuss how these scores may be parameterized and learned.  Many systems \cite{Cucerzan07,Milne2008,KulkarniSRC09} simply hardwire pairwise scores.

%In the \emph{single link} model, we compute these scores deterministically and use them as fixed model inputs. In the \emph{attention} model, we parameterize the scores and learn them using a labeled dataset.

Coherence models typically attempt to maximize a {\em global} objective function that assigns a score to each complete labeling ${\bf y} = (y_1,\ldots, y_n)$. 
%, given the local and pairwise scores. 
An example of such a function is the sum of all singleton and pairwise scores for each label:
 %A common example of such a function $g(y_1,\ldots,y_n)$ is:
\be
g({\bf y}) = \sum_i s_i(y_i) + \sum_i \sum_{j :  j \neq i} s_{ij}(y_i,y_j).
\label{eq:global_obj}
\ee 
One disadvantage of this approach is that maximizing $g$ corresponds to finding the MAP assignment of a general pairwise Markov random field, and is hence
NP hard for the general case \cite{wainwright2008graphical}. Another limitation is that non-salient entities may only have a few relations in a document, and summing over all mentions may dilute the evidence for such entities. In this paper we explore alternative objectives, relying on attention and tractable inference.



%%% Local Variables: ***
%%% mode:latex ***
%%% TeX-master: "main.tex"  ***
%%% tex-main-file: "main.tex"  ***
%%% End: ***
