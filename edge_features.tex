Edge features $\fp$ are set as follows:
\begin{itemize}
\item {\bf Gold link features:} Denote by $l(y_i,y_j)$ the number of links between $y_i$ and $y_j$ in Wikipedia (in either direction), capped to the value $15$. We introduce a set of $15$ features where the $j^{th}$ feature is set to $1$ if $l(y_i,y_j) = j$, and $0$ otherwise. This is meant to capture non-linear dependencies of coherence
on the number of links.
\item {\bf Self-trained link features:} Since Wikipedia links are quite sparse, we augment them by labeling all of Wikipedia with our baseline resolver. %a baseline entity tagger from \cite{Lazic2015}. 
Denote $r(y_i,y_j)$ the number of links in this labeled dataset. Then we add as features $\log{r(y_i,y_j)}$ and $\delta_{r(y_i,y_j),0}$. We use a log scale here since there are more links than in the gold case. 
\item {\bf Relation features:} Denote by $r(y_i,y_j)$ the number of existing relations between $y_i$ and $y_j$ in the knowledge base (FreeBase), capped to the value $15$ (in practice this is usually $0$ or $1$). Introduce a set of $15$ features where the $j^{th}$ feature is set to $1$ if $r(y_i,y_j) = j$, and $0$ otherwise. 
\end{itemize}
We chose this relatively small set of features, so that it can be trained from the small training sets of CoNLL and TAC (both have a few thousand ground truth mentions).