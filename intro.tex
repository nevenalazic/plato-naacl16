\section{Introduction}
\label{sec:intro}

Entity resolution (ER) is the task of mapping mentions of entities in
text to corresponding records in a knowledge base (KB)
\cite{BunescuP06,Cucerzan07,KulkarniSRC09,Dredze2010,Hoffart2011,Hachey2013130}.
ER is a challenging problem because mentions are often ambiguous on
their own, and can only be resolved given appropriate context.  For
example, the mention \qtext{Beirut} may refer to the capital of
Lebanon, the band from New Mexico, or a drinking game
(Figure~\ref{fig:ereg}).
Names may also refer to entities that are not in the KB, a problem
 known as \emph{{\NIL} detection}.
% ER can improve text classification \cite{Gabrilovich2007}, information
% extraction \cite{Lin2012}, coreference resolution
% \cite{finin2009Coreference,mayfield2009cross} and other
% language-processing tasks.
%Kwiatkowski2011groundedsemparsing

Most ER systems consist of a \emph{mention model}, a \emph{context
  model}, and a \emph{coherence
  model}~\cite{Milne2008,Cucerzan07,Ratinov11,Hoffart2011,Hachey2013130}.
The mention model associates each entity with its possible textual
representations (also known as aliases or surface forms).  The context
model helps resolve an ambiguous mention using textual features
extracted from the surrounding context. 
%, such as the enclosing sentence and salient noun phrases in the document. 
The coherence model, the focus of this work, encourages all mentions to resolve to
entities that are related to each other.  Relations may be established
via the KB, Web links, embeddings, or other resources.

Coherence models often define an objective function that includes
local and pairwise candidate scores, where the pairwise scores
correspond to some notion of coherence or relation
strength.\footnote{An exception to this framework are topic models in
  which a topic may generate both entities and words, e.g.,
  \cite{kataria2011,HanS12,houlsby2014scalable}.} 
  Support for a candidate is typically aggregated over
relations to all other entities in the document. One problem with this approach is that it may
dilute evidence for entities that are not salient in the document, or
not well-connected in the KB. This is an issue we aim to address.
%Finding an entity
% labeling that maximizes the objective is usually intractable and
% tackled via various approximations, which we discuss in more detail in
% Section~\ref{sec:related}. 



\begin{figure*}[!ht]
\centering
\begin{tikzpicture}[node distance=1.1cm and 2cm,>=stealth',auto]
  \tikzstyle{candidate}=[align=left,rectangle,rounded corners=2mm,thick,draw=black,fill=gray!10,minimum size=6mm,text width=52,font=\small]
  \tikzstyle{var}=[rectangle,thick,minimum size=6mm,font=\large]
  \tikzstyle{every label}=[black]

 \begin{scope}[-,line width=1pt]
    % First mention.
   \node [candidate] (c11){Beirut \\(city in Leb.)};
   \node [candidate] (c12) [below of=c11]  {Beirut \\(band)};
   \node [candidate] (c13)  [below of=c12] {Beirut \\(game)};
   \node [var] (y1) [above=0.2cm of c11] {$y_1$};

   % Second mention.
    \node [candidate] (c21) [right= of c11] {Santa Fe \\(city in NM)};
        \node [candidate] (c22) [below of=c21]  {Santa Fe \\ (film)};
    \node [candidate] (c23)  [below of=c22] {Santa Fe \\(city in Cuba)};
       \node [var] (y2) [above=0.2cm of c21] {$y_2$};
     
     % Third mention.
    \node [candidate] (c31) [right=of c21] {New Mexico \\(state)};
    \node [candidate] (c32)  [below of=c31] {New Mexico \\(university)};
    \node [candidate] (c33)  [below of=c32] {New Mexico\\ (ship)};
    \node [var] (y3) [above=0.2cm of c31] {$y_3$};
 
   \path (c12.east) edge (c21.west);
   \path (c12.east) edge [bend left=45] (c31.west);
   \path (c21.east) edge (c31.west);
   \path (c22.east) edge (c31.west);
   
  \end{scope}
  
  \begin{pgfonlayer}{background}
  \filldraw [fill=black!10,draw=black!10,line width=4mm]% [line width=4mm,black!10]
      (c11.north  -| c11.east)  rectangle (c13.south  -| c13.west)
      (c21.north -| c21.east) rectangle (c23.south -| c23.west)
      (c31.north  -| c31.east)  rectangle (c33.south  -| c33.west);
  \end{pgfonlayer}
\end{tikzpicture}
\caption{Illustration of the ER problem for three mentions ``Beirut'', ``New Mexico'' and ``Santa Fe''. each mention has three possible disambiguations. Edges link disambiguations that have Wikipedia links between their respective pages.}
\label{fig:ereg}
\end{figure*}



%The above coherence objectives consider the sum over all pairwise terms. Such
%``all-pairs'' coherence objectives aggregate over salient, well-connected
%entities and less prominent entities that are not so well-connected.
%In effect, all-pairs objectives seek evidence in favor of a candidate
%entity from \emph{all} other mentions in the document, evidence that
%may simply not exist for non-salient entities.  However, valuable
%support in favor of a candidate entity may be provided by a small
%number of entities chosen for other mentions.  Thus, all-pairs
%objectives risk losing these delicate signals.  Our primary goal is to
%fix this problem.

In this work, we introduce a novel coherence model with an attention mechanism, where the 
score for each candidate only depends on a small subset of mentions.
 Attention has recently been
used with considerable empirical success in tasks such as translation
\cite{bahdanau2014neural} and image caption generation
\cite{xu2015show}. We argue that attention is also desirable for
collective ER due to the discussed imbalance in the number of
relations for different entities.

Attention models typically have a single focus, implemented using the
 softmax function. Our model allows each candidate to
 focus on multiple mentions, and to implement it we introduce a 
 novel smooth version of the
 multi-focus attention function, which generalizes soft-max.
%Our model relies on a novel smooth version of the multi-focus
%attention function, which generalizes the single-focus softmax
%function. 
%We use a simple and efficient inference procedure, and show
%how the model parameters can be learned from data.

We augment a competitive local context model (which does not use
coherence) similar to \cite{Lazic2015} with our coherence
framework, leading to performance improvements on three \todo{or four} evaluation benchmarks:
CoNLL 2003 \cite{Hoffart2011} and TAC KBP 2010--2012 \cite{TAC2010,TAC2011,TAC2012}.
%On the CoNLL 2003 dataset \cite{Hoffart2011}, we see a relative
%reduction in error (RRIE) of \todo{FILL\%}.  On the TAC KBP 2010--2012
% datasets, we get a RRIE of in-KG entities around \todo{FILL\%}.

%Our contributions thus consist of defining a novel multi-focal
%attention model and applying it
%successfully to an entity resolution system.  
\secref{sec:notation} introduces notation.
\secref{sec:attention} presents the new multi-focus attention
model.  \secref{sec:maxsum} discusses a baseline model with
single focus.  \secref{sec:related} reviews related work.
\secref{sec:expt} presents experimental results.
%Our premise is that this may also hold for entity relations:
%aggregating support for an entity label over the whole document may
%dilute the evidence for non-salient entities. We explore two new
%approaches to coherence that focus on a \emph{limited number} of
%relations for each candidate, rather than relations to all other
%entities.
%\begin{figure*}[!ht]
\begin{tikzpicture}[node distance=1.5cm,>=stealth',bend angle=35,auto]
  \tikzstyle{candidate}=[rectangle,dashed,rounded corners=2mm,thick,draw=black,fill=white,minimum size=6mm]
  \tikzstyle{selected} = [rectangle,rounded corners=2mm,thick,draw=black,fill=gray!10,minimum size=6mm]
  \tikzstyle{mention}=[rectangle,thick,draw=black!75,
  			  fill=black!10,minimum size=6mm]
  \tikzstyle{every label}=[black]

 \begin{scope}
    % First mention.
   % \node [mention] (m1){\qtext{Beirut}};
   % \node [candidate] (c11) [label=above:0.5] {Beirut, Lebanon};
   \node [candidate] (c11) {Beirut, Lebanon};
    \node [candidate] (c12) [below of=c11]  {Beirut (game)};
    \node [selected] (c13)  [below of=c12] {Beirut (band)};

   % Second mention.
   % \node [mention] [right=3cm of m1](m2){\qtext{Santa Fe}};
    \node [selected] (c21) [right=2cm of c11] {Santa Fe, New Mexico};
        \node [candidate] (c22) [below of=c21]  {Santa Fe (1951 Western)};
    \node [candidate] (c23)  [below of=c22] {Santa Fe (Beirut song)};

      
     % Third mention.
   % \node [mention] [right=3cm of m2](m3){\qtext{New Mexico}};
    \node [selected] (c31) [right=2cm of c21] {New Mexico (state)};
    \node [candidate] (c32)  [below of=c31] {Univ. New Mexico};
 
      
   %\path (c13) edge [post,line width=2pt] node[above]{0.5} (c21);
   \path (c13) edge [post,line width=2pt] (c21);
   \path (c13) edge [post,dashed,line width=2pt] (c23);
   \path (c31) edge [post,line width=2pt] (c21);
   \path (c21) edge [post,line width=2pt,bend right] (c31);
  \end{scope}
\end{tikzpicture}
\caption{Example coherency graph for mentions $\qtext{Beirut}$, $\qtext{Santa Fe}$ and $\qtext{New Mexico}$. Solid lines indicate a valid solution subgraph, where each selected candidate has at most one outgoing edge to another candidate. TODO: better example?}
\label{fig:graph}
\end{figure*}


%  Accordingly, we propose to choose the label for each mention based on the best support from a \emph{limited number} of other mentions.  In other words,  each mention is labeled by (tractable) inference in a star graph, but one where most edges are (dynamically) ignored.

\comment{
\subsection{Our contributions}
\label{sec:intro:our}
We use these coherence models to re-rank candidates generated by Plato \cite{Lazic2015}, a recent entity resolution system that has highly competitive performance and does not include a coherence component. This leads to performance improvements on three benchmarks, and yields new state-of-the-art results on the TAC KBP 2011 and 2012 datasets.
}






%%% Local Variables: ***
%%% mode:latex ***
%%% TeX-master: "main.tex"  ***
%%% tex-main-file: "main.tex"  ***
%%% End: ***
